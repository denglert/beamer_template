%
% custom definitions
%

\newcommand {\roots}    {\ensuremath{\sqrt{s}}}
\newcommand {\rootsNN}  {\ensuremath{\sqrt{s_{_{NN}}}}}
\newcommand {\dndy}     {\ensuremath{dN/dy}}
\newcommand {\dnchdy}   {\ensuremath{dN_{\mathrm{ch}}/dy}}
\newcommand {\dndeta}   {\ensuremath{dN/d\eta}}
\newcommand {\dnchdeta} {\ensuremath{dN_{\mathrm{ch}}/d\eta}}
\newcommand {\dndpt}    {\ensuremath{dN/d\pt}}
\newcommand {\dnchdpt}  {\ensuremath{dN_{\mathrm{ch}}/d\pt}}
\newcommand {\deta}     {\ensuremath{\Delta\eta}}
\newcommand {\dphi}     {\ensuremath{\Delta\phi}}
\renewcommand {\pt}       {\ensuremath{p_\mathrm{T}}}
\newcommand {\pttrg}       {\ensuremath{p_\mathrm{T}^{\mathrm{trig}}}}
\newcommand {\ptass}       {\ensuremath{p_\mathrm{T}^{\mathrm{assoc}}}}
\newcommand {\ptref}       {\ensuremath{p_\mathrm{T}^{\mathrm{ref}}}}
\newcommand {\AJ}       {\ensuremath{A_J}}
\newcommand {\npart}    {\ensuremath{\mathrm{N}_\mathrm{part}}}
\newcommand {\ncoll}    {\ensuremath{\mathrm{N}_\mathrm{coll}}}
\newcommand {\dEdx} {\ensuremath{\frac{dE}{dx}}}
%\newcommand {\ET}       {\ensuremath{E_T}}  %% already defined in ../../../utils/trunk/general/ptdr-definitions.tex
\newcommand {\llangle}       {\ensuremath{\langle\langle}}
\newcommand {\rrangle}       {\ensuremath{\rangle\rangle}}
\newcommand {\kshort}    {\ensuremath{K^{0}_{\mathrm{s}}}}
\newcommand {\vsecsig}    {\ensuremath{v_2^\mathrm{signal}}}
\newcommand {\vsecbkg}    {\ensuremath{v_2^\mathrm{bkg}}}
\newcommand {\vsecobs}    {\ensuremath{v_2^\mathrm{obs}}}
\newcommand {\vtrdsig}    {\ensuremath{v_3^\mathrm{signal}}}
\newcommand {\vtrdbkg}    {\ensuremath{v_3^\mathrm{bkg}}}
\newcommand {\vtrdobs}    {\ensuremath{v_3^\mathrm{obs}}}
\newcommand {\vnmass}    {\ensuremath{v_n^\mathrm{massband}}}
\newcommand {\vnside}    {\ensuremath{v_n^\mathrm{sideband}}}
\newcommand {\vntrue}    {\ensuremath{v_n^\mathrm{true \Pphi s}}}
\newcommand {\Vntrgass} {\ensuremath{V_{n\Delta} (\pttrg,\ptass)}}
\newcommand {\Vnassass} {\ensuremath{V_{n\Delta} (\ptass,\ptass)}}
\newcommand {\vnass}    {\ensuremath{v_n (\ptass)}}
\newcommand {\vntrg}    {\ensuremath{v_n (\pttrg)}}
\newcommand{\ket}       {\ensuremath{KE_{\mathrm{T}}}\xspace}
\newcommand{\etadom} {\ensuremath{\eta_{d}}}
\newcommand{\zvtx}  {\ensuremath{z_{\rm vtx}}}

\newcommand {\pp}    {\mbox{pp}}
\newcommand {\ppbar} {\mbox{p\={p}}}
\newcommand {\pbarp} {\mbox{p\={p}}}
\newcommand {\PbPb}  {\mbox{PbPb}}
\newcommand {\AuAu}  {\mbox{AuAu}}
\newcommand {\pPb}  {\ensuremath{\text{pPb}}\xspace}
\newcommand {\AonA}  {\ensuremath{\text{AA}}\xspace}
%\newcommand {\pA}    {\ensuremath{\text{pA}}\xspace}
\newcommand {\dAu}    {\ensuremath{\text{dAu}}\xspace}
\newcommand{\non}    {\ensuremath{N_\mathrm{trk}^\mathrm{online}}\xspace}
\newcommand{\noff}    {\ensuremath{N_\mathrm{trk}^\mathrm{offline}}\xspace}
\newcommand{\npri}    {\ensuremath{N_\mathrm{trk}^\mathrm{best}}\xspace}
\newcommand{\nsec}    {\ensuremath{N_\mathrm{trk}^\mathrm{add}}\xspace}

\newcommand{\m}{\ensuremath{\,\text{m}}\xspace}
\newcommand {\naive}    {na\"{\i}ve}
\providecommand{\GEANT} {{Geant}\xspace}
\providecommand{\PHOJET} {\textsc{phojet}\xspace}
\providecommand{\PYNEW} {\textsc{pythia8}\xspace}

% roman d for derivatives
\def\d{\mathrm{d}}

% pennames
\providecommand{\Pgp}{\ensuremath{\mathrm{\pi}}}
\providecommand{\PK}{\ensuremath{\mathrm{K}}}
\providecommand{\Pp}{\ensuremath{\mathrm{p}}}
\providecommand{\PKzS}{\ensuremath{\mathrm{K^0_S}}}
\providecommand{\Pp}{\ensuremath{\mathrm{p}}}
\providecommand{\Pap}{\ensuremath{\mathrm{\overline{p}}}}
\providecommand{\PgL}{\ensuremath{\Lambda}}
\providecommand{\PagL}{\ensuremath{\overline{\Lambda}}}
\providecommand{\PgS}{\ensuremath{\Sigma}}
\providecommand{\PgSm}{\ensuremath{\Sigma^-}}
\providecommand{\PgSp}{\ensuremath{\Sigma^+}}
\providecommand{\PagSm}{\ensuremath{\overline{\Sigma}^-}}
\providecommand{\PagSp}{\ensuremath{\overline{\Sigma}^+}}
\providecommand{\Pphim}{\varphi}
%\renewcommand{\GEANTfour} {{Geant4}\xspace}


\newcommand {\photonjet}{photon+jet}
\let\gammajet\photonjet

\providecommand{\ptjet}{\ptj}
\providecommand{\ptgamma}{\ptg}
\providecommand{\ptphoton}{\ptg}
\providecommand{\rjg}{\ensuremath{R_{J\cPgg}}\xspace}
\providecommand{\xjg}{\ensuremath{x_{J\cPgg}}\xspace}
\providecommand{\sjg}{\ensuremath{\sigma(\Delta\phi_{J\cPgg})}\xspace}
\providecommand{\ptg}{\ensuremath{p_{\mathrm{T},\cPgg}}\xspace}
\providecommand{\dphijg}{\ensuremath{\Delta\phi_{J\cPgg}}\xspace}
\providecommand{\avexjg}{\ensuremath{\langle x_{J\cPgg}\rangle}\xspace}
\providecommand{\sigeta}{\ensuremath{\sigma_{\eta\eta}}\xspace}
\providecommand{\HYDJET} {\textsc{hydjet}\xspace}
\providecommand{\PYTHIAHYDJET} {\textsc{pythia + hydjet}\xspace}
\providecommand{\PYTHIA} {\textsc{pythia}\xspace}
\providecommand{\HIJING} {\textsc{hijing}\xspace}
\providecommand{\AMPT} {\textsc{ampt}\xspace}
\providecommand{\EPOS} {\textsc{epos lhc}\xspace}
\providecommand{\mubinv} {\ensuremath{\,\mu\mathrm{b}^{-1}}\xspace}


% Correl2D figure caption
\newcommand{\correltwoDFigsCaption}[6]{
			 The 2D two-particle correlation functions for pion, kaon and protons in the first,
			 second and third rows respectively with {#1} < \pttrg\ < {#2} GeV/c
			 and {#3} < \ptass\ < {#4} GeV/c in \pPb\ collisions at \rootsNN\ =
			 5.02\TeV . The multiplicity is \noff = [{#5}] for the first coloumn
  			 and \noff = [{#6}] for the second coloumn.}

